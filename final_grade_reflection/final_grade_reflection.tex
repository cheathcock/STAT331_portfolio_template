% Options for packages loaded elsewhere
\PassOptionsToPackage{unicode}{hyperref}
\PassOptionsToPackage{hyphens}{url}
\PassOptionsToPackage{dvipsnames,svgnames,x11names}{xcolor}
%
\documentclass[
  letterpaper,
  DIV=11,
  numbers=noendperiod]{scrartcl}

\usepackage{amsmath,amssymb}
\usepackage{lmodern}
\usepackage{iftex}
\ifPDFTeX
  \usepackage[T1]{fontenc}
  \usepackage[utf8]{inputenc}
  \usepackage{textcomp} % provide euro and other symbols
\else % if luatex or xetex
  \usepackage{unicode-math}
  \defaultfontfeatures{Scale=MatchLowercase}
  \defaultfontfeatures[\rmfamily]{Ligatures=TeX,Scale=1}
\fi
% Use upquote if available, for straight quotes in verbatim environments
\IfFileExists{upquote.sty}{\usepackage{upquote}}{}
\IfFileExists{microtype.sty}{% use microtype if available
  \usepackage[]{microtype}
  \UseMicrotypeSet[protrusion]{basicmath} % disable protrusion for tt fonts
}{}
\makeatletter
\@ifundefined{KOMAClassName}{% if non-KOMA class
  \IfFileExists{parskip.sty}{%
    \usepackage{parskip}
  }{% else
    \setlength{\parindent}{0pt}
    \setlength{\parskip}{6pt plus 2pt minus 1pt}}
}{% if KOMA class
  \KOMAoptions{parskip=half}}
\makeatother
\usepackage{xcolor}
\setlength{\emergencystretch}{3em} % prevent overfull lines
\setcounter{secnumdepth}{-\maxdimen} % remove section numbering
% Make \paragraph and \subparagraph free-standing
\ifx\paragraph\undefined\else
  \let\oldparagraph\paragraph
  \renewcommand{\paragraph}[1]{\oldparagraph{#1}\mbox{}}
\fi
\ifx\subparagraph\undefined\else
  \let\oldsubparagraph\subparagraph
  \renewcommand{\subparagraph}[1]{\oldsubparagraph{#1}\mbox{}}
\fi


\providecommand{\tightlist}{%
  \setlength{\itemsep}{0pt}\setlength{\parskip}{0pt}}\usepackage{longtable,booktabs,array}
\usepackage{calc} % for calculating minipage widths
% Correct order of tables after \paragraph or \subparagraph
\usepackage{etoolbox}
\makeatletter
\patchcmd\longtable{\par}{\if@noskipsec\mbox{}\fi\par}{}{}
\makeatother
% Allow footnotes in longtable head/foot
\IfFileExists{footnotehyper.sty}{\usepackage{footnotehyper}}{\usepackage{footnote}}
\makesavenoteenv{longtable}
\usepackage{graphicx}
\makeatletter
\def\maxwidth{\ifdim\Gin@nat@width>\linewidth\linewidth\else\Gin@nat@width\fi}
\def\maxheight{\ifdim\Gin@nat@height>\textheight\textheight\else\Gin@nat@height\fi}
\makeatother
% Scale images if necessary, so that they will not overflow the page
% margins by default, and it is still possible to overwrite the defaults
% using explicit options in \includegraphics[width, height, ...]{}
\setkeys{Gin}{width=\maxwidth,height=\maxheight,keepaspectratio}
% Set default figure placement to htbp
\makeatletter
\def\fps@figure{htbp}
\makeatother

\KOMAoption{captions}{tableheading}
\makeatletter
\makeatother
\makeatletter
\makeatother
\makeatletter
\@ifpackageloaded{caption}{}{\usepackage{caption}}
\AtBeginDocument{%
\ifdefined\contentsname
  \renewcommand*\contentsname{Table of contents}
\else
  \newcommand\contentsname{Table of contents}
\fi
\ifdefined\listfigurename
  \renewcommand*\listfigurename{List of Figures}
\else
  \newcommand\listfigurename{List of Figures}
\fi
\ifdefined\listtablename
  \renewcommand*\listtablename{List of Tables}
\else
  \newcommand\listtablename{List of Tables}
\fi
\ifdefined\figurename
  \renewcommand*\figurename{Figure}
\else
  \newcommand\figurename{Figure}
\fi
\ifdefined\tablename
  \renewcommand*\tablename{Table}
\else
  \newcommand\tablename{Table}
\fi
}
\@ifpackageloaded{float}{}{\usepackage{float}}
\floatstyle{ruled}
\@ifundefined{c@chapter}{\newfloat{codelisting}{h}{lop}}{\newfloat{codelisting}{h}{lop}[chapter]}
\floatname{codelisting}{Listing}
\newcommand*\listoflistings{\listof{codelisting}{List of Listings}}
\makeatother
\makeatletter
\@ifpackageloaded{caption}{}{\usepackage{caption}}
\@ifpackageloaded{subcaption}{}{\usepackage{subcaption}}
\makeatother
\makeatletter
\@ifpackageloaded{tcolorbox}{}{\usepackage[many]{tcolorbox}}
\makeatother
\makeatletter
\@ifundefined{shadecolor}{\definecolor{shadecolor}{rgb}{.97, .97, .97}}
\makeatother
\makeatletter
\makeatother
\ifLuaTeX
  \usepackage{selnolig}  % disable illegal ligatures
\fi
\IfFileExists{bookmark.sty}{\usepackage{bookmark}}{\usepackage{hyperref}}
\IfFileExists{xurl.sty}{\usepackage{xurl}}{} % add URL line breaks if available
\urlstyle{same} % disable monospaced font for URLs
\hypersetup{
  pdftitle={Final Grade Reflection},
  pdfauthor={Camille Heathcock},
  colorlinks=true,
  linkcolor={blue},
  filecolor={Maroon},
  citecolor={Blue},
  urlcolor={Blue},
  pdfcreator={LaTeX via pandoc}}

\title{Final Grade Reflection}
\author{Camille Heathcock}
\date{}

\begin{document}
\maketitle
\ifdefined\Shaded\renewenvironment{Shaded}{\begin{tcolorbox}[breakable, enhanced, frame hidden, boxrule=0pt, borderline west={3pt}{0pt}{shadecolor}, sharp corners, interior hidden]}{\end{tcolorbox}}\fi

In this document, you make a data-based argument for the grade you've
earned in this course. Your argument should include evidence from the
supporting artifacts you've provided.

\hypertarget{the-output-document-should-be-a-pdf-or-a-word-document-as-it-should-be-a-maximum-of-2-pages.}{%
\subsection{\texorpdfstring{The output document should be a PDF or a
Word Document, as it should be a \textbf{maximum} of
2-pages.}{The output document should be a PDF or a Word Document, as it should be a maximum of 2-pages.}}\label{the-output-document-should-be-a-pdf-or-a-word-document-as-it-should-be-a-maximum-of-2-pages.}}

\textbf{How have you demonstrated a commitment to continued learning?}

I have demonstrated a commitment to continued learning through my
dedication to understand my past mistakes. I spend a fair amount of time
completing the initial assignments for this class, but oftentimes I am
not able to perfect them the first time around. However, I think there
is so much value in receiving feedback from a coding expert (Dr.~T), and
developing my initial attempt at a solution. I really appreciate that I
am able to build on my initial encoding of the class curriculum by going
back to my past work and revising my thinking. This allows my brain to
reinforce the concepts we have learned, as well as encourages me to tie
different concepts from different weeks together within my code.
Additionally, I am able to reinforce class concepts within the structure
of class and office hours. Oftentimes, I will learn a concept but not
fully understand it until I see how it is used in context. The in-person
meetings allow me to develop my thinking and really let the concepts
click. I appreciate that we have many opportunities to learn and relearn
the theories that are relevant for the course, and apply them in various
contexts. All in all, I have demonstrated a commitment to continued
learning through my class participation and revision process.

\textbf{How have you grown as a team member and leader?}

I have grown as a team member through my communication and collaboration
skills. I often get stuck on a tiny part of a given assignment-- whether
it be a syntax error or otherwise, it is very frustrating. I have
developed various troubleshooting techniques for when I encounter this
kind of issue, like using the help file and stack overflow, as well as
checking the discord channel. However, it is nice to have a team as a
support system for small issues like this. It is always good to have a
fresh eye look at something I have been grueling over, as I am often
missing an obvious flaw in the code. I have become much more comfortable
communicating with my team about various assignments and issues.
Additionally, I am always happy to help out my team members if they are
confused about something that I understand. In terms of the ways I have
grown within my team, I think I have developed as a team member and
leader. I always try to make sure the members of our team are on the
same page, and facilitate a group conversation if it needs to be had.

\textbf{How have you contributed to creating a respectful classroom
learning community?}

I have contributed to creating a respectful classroom learning community
through my participation and engagement during class. I really
appreciate the dynamic of the STAT 331 classroom, because I feel very
comfortable asking questions, as well as collaborating with the people
around me. It is a very inclusive and low-stakes environment, so I feel
quite comfortable participating in class discussions. I also appreciate
that the whole class works on the same assignment at once, because I
will often listen in on questions my fellow classmates have asked, which
can be extremely helpful.

\textbf{Are you accomplishing the goals you outlined for yourself at the
start of the course?}

The goals I outlined at the start of the course were: To thoroughly
understand what is expected from me and keep up with the workload, and
to be able to write R code that is not only useful for all sorts of
applications, but also allows me to express creativity. Unfortunately, I
don't think I have fully achieved the first goal. I feel as though I
understand what is expected of me and I understand the flow and workload
of the class. However, I have not been the best at keeping up with all
of our assignments. Around week 4-5, I got quite behind on the
coursework. This was quite frustrating when attempting activities in
class, as I felt very behind and lost. However, I am proud of myself for
rebounding from this self-imposed setback and catching up the following
weekend. Instead of wallowing in my stressed state of being, I dedicated
time to learning the concepts that I had not grasped. My goal for the
rest of the quarter is to keep up with the outlined concepts and stay on
top of the workload. My second goal outlined at the start of the course
was to creatively write R code that is useful in all sorts of
applications. As I said in my last reflection, this goal is a broad one.
I can't say I have fully achieved this, as I think I have a lot of room
for improvement as a programmer. However, I think I am working in the
right direction. Even since I wrote the last reflection, I feel as
though I have developed my coding skills a lot and am able to do many
more of the learning targets. My new developed goal for the last few
weeks of the quarter is to get better at being creative within the
context of a particular assignment, especially when given creative
freedom. I struggled on this in Challenge 4, when we were tasked with
creating a useful application of a dataset that we found. I hope to
improve my skills in this area by my final reflection. To conclude,
although I have not fully achieved the goals I outlined for myself, I
feel confident in my ability to achieve them by the end of the quarter.
I feel as though I deserve a B+ for my work in the class. I don't think
I have quite crossed the threshold into the A range, especially since I
was behind for a little while. However, I think I have demonstrated
proficiency for many of the learning targets, and hopefully it will be a
majority of the learning targets by the end of the quarter. In terms of
revising my thinking, I think I have regularly learned from my
revisions, which has impacted my code in subsequent assignments, and I
think I have generally revised all of the assignments when the
opportunity is given. I have also extended my thinking by completing
most of the challenge assignments and challenging myself within the
context of these problems. I hope to challenge myself even more
throughout the remainder of the quarter. I also think I have been a
great team member. I am good at encouraging communication within my team
and will ask Dr.~T. questions when necessary. I am always present and
cooperative, I always fill my role in the team, and I always try to be
as respectful as possible. However, since I was behind for a little
while, I am more in the B range for being prepared for class each week,
as well as completing assignments and quizzes on time. In terms of
peer-reviewing code, I think I have been responsive to constructive
criticism, and have put my best effort into reviewing my peers' code.
For the code I have reviewed, I have given constructive criticism with
kindness, as well as praised good qualities of other's code. However, I
am more in the B range for this requirement, since I have not been able
to consistently complete peer-reviews for assignments I have turned in
late.



\end{document}
